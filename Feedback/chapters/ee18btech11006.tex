
\begin{enumerate}[label=\thesection.\arabic*.,ref=\thesection.\theenumi]
\numberwithin{equation}{enumi}

\item An op amp having a low-frequency gain of $10^{3}$ and a single-pole rolloff at $10^{4}$ rad/s is connected in a negative feedback loop via a feedback network having a transmission k and a two-pole rolloff at $10^{4}$ rad/s. Find the value of k above which the closed-loop amplifier becomes unstable.


\solution 
\begin{figure}[!ht]
	\begin{center}
		\resizebox{\columnwidth}{!}{\tikzstyle{block} = [draw, fill=blue!20, rectangle, 
    minimum height=3em, minimum width=6em]
\tikzstyle{sum} = [draw, fill=blue!20, circle, node distance=1cm]
\tikzstyle{input} = [coordinate]
\tikzstyle{output} = [coordinate]
\tikzstyle{pinstyle} = [pin edge={to-,thin,black}]

\begin{tikzpicture}[auto, node distance=2cm,>=latex']
    \node [input, name=input] {};
    \node [sum, right of=input] (sum) {};
    \node [block, right of=sum] (controller) {$G$};
    \node [output, right of=controller] (output) {};
    \node [block, below of=controller] (feedback) {$H$};
    \draw [draw,->] (input) -- node {} (sum);
    \draw [->] (sum) -- node {$V_i$} (controller);
    \draw [->] (controller) -- node [name=y] {$V_o$}(output);
    \draw [->] (y) |- (feedback);
    \draw [->] (feedback) -| node[pos=0.99]{$-$}  node [near end] {$V_f$} (sum);
\end{tikzpicture}
}
	\end{center}
\caption{}
\label{fig:ee18btech11006_1}
\end{figure} \\
The given oscillator has a low frequency gain $10^3$ and a single-pole rolloff at $10^4$ rad/s. So we have a open loop amplifier gain 
\begin{align}
G(s)&= \frac{10^3}{1+\frac{s}{10^4}}
\end{align}
Given feedback network transmission gain k and two-pole rolloff at $10^4$ rad/s. So, the feedback factor would be
\begin{align}
H(s)&= \frac{k}{\left(1+\frac{s}{10^4}\right)^2}   
\end{align}
The resulting loop-gain would be 
\begin{align}
G(s)H(s) &= \frac{10^3k}{\left(1+\frac{s}{10^4}\right)^3}
\end{align}
\begin{table}[!ht]
\centering
\tikzstyle{block} = [draw, fill=blue!20, rectangle, 
    minimum height=3em, minimum width=6em]
\tikzstyle{sum} = [draw, fill=blue!20, circle, node distance=1cm]
\tikzstyle{input} = [coordinate]
\tikzstyle{output} = [coordinate]
\tikzstyle{pinstyle} = [pin edge={to-,thin,black}]

\begin{tikzpicture}[auto, node distance=2cm,>=latex']
    \node [input, name=input] {};
    \node [sum, right of=input] (sum) {};
    \node [block, right of=sum] (controller) {$G$};
    \node [output, right of=controller] (output) {};
    \node [block, below of=controller] (feedback) {$H$};
    \draw [draw,->] (input) -- node {} (sum);
    \draw [->] (sum) -- node {$V_i$} (controller);
    \draw [->] (controller) -- node [name=y] {$V_o$}(output);
    \draw [->] (y) |- (feedback);
    \draw [->] (feedback) -| node[pos=0.99]{$-$}  node [near end] {$V_f$} (sum);
\end{tikzpicture}

\caption{}
\label{table:ee18btech11006_Factors}
\end{table}\\
The closed loop gain would be
\begin{align}
T(s)=\frac{G(s)}{1+G(s)H(s)}
\end{align}
Generalized condition for the system to be stable:
\begin{align}
T(j\omega)=\frac{G(j\omega)}{1+G(j\omega)H(j\omega)}
\end{align}
Loop gain, $L(j\omega)=G(j\omega)H(j\omega)$.
\begin{align}
L(j\omega)=\abs{{G(j\omega)H(j\omega)}}e^{j\phi(\omega)}
\end{align}
Let the frequency at which phase angle $\phi(\omega)$ becomes $180^{\circ}$ be $\omega_{180}$. At $\omega = \omega_{180}$ , $L(j\omega)$ is a negative real number.
\begin{itemize}
    \item if $\abs{{G(j\omega_{180})H(j\omega_{180})}}$ $<$ 1,$T(j\omega_{180}) >$ 1 \\ $\implies$ 
    system is stable.
    \item if $\abs{{G(j\omega_{180})H(j\omega_{180})}}$ $=$ 1,$T(j\omega_{180}) = \infty $ \\ $\implies$
    system is unstable.
    \item if $\abs{{G(j\omega_{180})H(j\omega_{180})}}$ $>$ 1,$T(j\omega_{180}) <$ 1 \\ $\implies$ 
    system is unstable.
\end{itemize}
For the given system :
\begin{align}
   \angle G(j\omega)H(j\omega) =  \angle\frac{10^3k}{\left(1+\frac{j\omega}{10^4}\right)^3} 
   = -3tan^{-1}\left({\frac{\omega}{10^4}}\right)
\end{align}
So,
\begin{align}
   180^{\circ}=  -3tan^{-1}\left({\frac{\omega_{180}}{10^4}}\right)\\
   \implies \omega_{180} = -\sqrt{3}\times 10^4
\end{align}
The Loop gain at $\omega_{180}$ is $G(j\omega_{180})H(j\omega_{180})$.
 The system becomes unstable if 
\begin{align}
G(j\omega_{180})H(j\omega_{180})\geq 1 \\
\implies \abs{\frac{10^3k}{\left(1+\frac{j\omega}{10^4}\right)^3}} \geq 1\\
\abs{\frac{10^3k}{\left(1-\sqrt{3}j\right)^3} } \geq 1
\end{align}
\begin{align}
     \frac{10^3k}{\abs{\sqrt{1+{\sqrt{3}}^2}}} \geq 1\\
     \frac{10^3k}{8} \geq 1 \\
     \implies k \geq 0.008
\end{align}
Hence, the value of k above which the system becomes unstable is 0.008. 
\item Design the circuit for the given parameters
\solution Consider the open loop gain,G \newline
\begin{figure}[!ht]
	\begin{center}
		\resizebox{\columnwidth}{!}{\begin{circuitikz}[american]
\usetikzlibrary{positioning, fit, calc}
\draw (0,0)to [open,v=$V_i$]++(0,-2)to[short]++(2,0)
(0,0)to++(2,0);
\draw (2,-1)node[draw,minimum width=2cm,minimum height=2cm] (load) {G}(2,0)
(2,0) -- (4,0);
\draw (2,-2)to [short](4,-2)
node at (4,-1.7) {$-$}
node at (4,-0.3){$+$}
node at (4,-1){$V_o$}
;
\end{circuitikz}
}
	\end{center}
\caption{}
\label{fig:ee18btech11006_4}
\end{figure} \\
Now, 
\begin{align}
V_i=V_{in}\\
G(s)=\frac{V_0}{V_i-V_-}= \frac{10^3}{1+\frac{s}{10^4}}
\end{align}
This is of the form $10^3 \left(\frac{R}{R+sL}\right)$.
This can be realized using the circuit:
\begin{figure}[!ht]
	\begin{center}
		\resizebox{\columnwidth}{!}{\begin{circuitikz}
\ctikzset{bipoles/length=1cm}

\draw 
(0, 0) node[op amp] (opamp) {}
(opamp.-)   to (-2, 0.35) node[ground]{}
(opamp.center) node[]{$10^3$}
node at (3.4,-1.2){$V_0$}
node at (3.4,0){$+$}
node at (3.4,-2.4){$-$}
(opamp.+) -- (-0.6,-0.35) to (-2,-0.35) to[V=$V_i$] (-2,-2.4) node[ground]{}
(opamp.out) to [L={$L_2$}](2.5,0) to [R=$R_2$](2.5,-2.4) node[ground]{}
;\end{circuitikz}
}
	\end{center}
\caption{}
\label{fig:ee18btech11006_3}
\end{figure} \\
Where the gain of the Op-Amp is $10^3$.
\begin{align}
    \frac{R_2}{L_2}=10^4
\end{align}
One set of values that would satisfy this condition would be 
\begin{align}
    R_2&= 100\ohm \\
    L_2&=1 \mu F
\end{align}
Consider the feedback factor, H \newline
\begin{figure}[!ht]
	\begin{center}
		\resizebox{\columnwidth}{!}{\begin{circuitikz}[american]
\usetikzlibrary{positioning, fit, calc}
\draw (0,0)to [open,v=$V_0$]++(0,-2)to[short]++(2,0)
(0,0)to++(2,0);
\draw (2,-1)node[draw,minimum width=2cm,minimum height=2cm] (load) {H}(2,0)
(2,0) -- (4,0);
\draw (2,-2) to[short] (4,-2)
node at (4,-1.7) {$-$}
node at (4,-0.3){$+$}
node at (4,-1){$V_f$}
;
\end{circuitikz}
}
	\end{center}
\caption{}
\label{fig:ee18btech11006_5}
\end{figure} \\
Now, 
\begin{align}
H(s)&=\frac{V_f}{V_0}= \frac{k}{\left(1+\frac{s}{10^4}\right)^2} = k\left({\frac{1}{1+\frac{2s}{10^4}+\frac{s^2}{10^8}}}\right)  
\end{align}
This is of the form,
\begin{align}
k\left(\frac{1}{1+s{R_1}{C_1}+{s^2}{L_1}{C_1}}\right) =  k\left(\frac{\frac{1}{s{C_1}}}{{R_1}+s{L_1}+\frac{1}{s{C_1}}}\right)  
\end{align}
This can be realized using the circuit,
\begin{figure}[!ht]
	\begin{center}
		\resizebox{\columnwidth}{!}{\begin{circuitikz}
\ctikzset{bipoles/length=1cm}

\draw 
(0, 0) node[op amp] (opamp) {}
(opamp.-) -- (-2,0.35)  node[ground]{}
(opamp.out) to[L =$L_1$,*-*] (2,0)to[R=$R_1$](4,0) to [C](4,-2.4) node[ground]{}
(opamp.+)   to (-1.8,-0.35)node at (-2,-0.35){$V_0$}
(opamp.center) node[]{k};
\draw node at (4.6,-1.2){$V_f$}
node at (4.6,0){$+$}
node at (4.6,-2.4){$-$}
node at (3.5,-1.2){$C_1$}
;\end{circuitikz}
}
	\end{center}
\caption{}
\label{fig:ee18btech11006_2}
\end{figure} \\
A set of values that satisfy these equations are,
\begin{align}
R_1&= 200\ohm \\
C_1&= 1\mu F\\
L_1&= 10mH 
\end{align}
The final circuit would be,
\begin{figure}[!ht]
	\begin{center}
		\resizebox{2\columnwidth}{!}{\begin{circuitikz}
\ctikzset{bipoles/length=1cm}
\draw 
(0, 0) node[op amp] (opamp) {}
(opamp.-) to (-1,0.35) node[ground]{}
(opamp.center) node{$k$}
(opamp.out) to [R=$R_1$](3,0) to [L=$L_1$](5,0) to [C=$C_1$](5,-3) node[ground]{};
\draw (7.5,-0.35) node[op amp] (opamp_1){}
(opamp_1.-) -- (5,0)
(opamp_1.+) to[V=$V_i$](6.7,-3) node[ground]{}
(opamp_1.center) node{$10^3$}
(opamp_1.out) to [L=$L_2$](10,-0.35) to [R=$R_2$](10,-3)node[ground]{}
(10,-0.35) -- (11,-0.35)--(11,-4)--(-0.8,-4)to (opamp.+)
(11,-0.35) to (11.5,-0.35)node at (11.8,-0.35){$V_0$}
(5,0)--(5,0.2)node at (5,0.5){$V_f$};
\end{circuitikz}
}
	\end{center}
\caption{}
\label{fig:ee18btech11006_6}
\end{figure}
\end{enumerate}
